\documentclass[a4paper,12pt]{article}
\usepackage{graphicx} % For images
\usepackage{listings} % For code
\usepackage{color}    % For syntax highlighting
\usepackage{amsmath}  % For math symbols (if needed)
\usepackage{float}

% Title and Author
\title{Practical Work 1: TCP File Transfer}
\author{Le Duy Anh}
\date{\ 26/11/2024}

\begin{document}
\maketitle

\section{Introduction}
This project uses Python sockets to implement a 1-to-1 file transfer system over TCP/IP. 
The system allows a client to send files to a server while ensuring integrity via checksum validation.
Enhancements include progress tracking and error handling, making it robust for large files.

\section{Protocol Design}
The file transfer protocol includes the following steps:
\begin{enumerate}
    \item The client connects to the server.
    \item The client sends file metadata (name and size).
    \item The server acknowledges and prepares to receive data.
    \item The client sends the file in chunks.
    \item The server verifies file integrity using a checksum.
\end{enumerate}

\begin{figure}[H]
    \centering
    \includegraphics[width=0.8\textwidth]{tcp_diagram.png}
    \caption{Protocol Flow for File Transfer}
\end{figure}

\section{System Organization}
The system uses a client-server model, where the server listens for incoming connections, 
and the client initiates the file transfer.

\section{Implementation}
The project uses Python’s socket library for communication. Key features include:
\begin{itemize}
    \item File transfer in chunks to support large files.
    \item MD5 checksum for file integrity verification.
    \item Progress tracking using the \texttt{tqdm} library.
    \item Error handling for network and file issues.
\end{itemize}

\subsection{Server Code Snippet}

\subsubsection{Receiving File Metadata}
\begin{figure}[H]
    \centering
    \includegraphics[width=0.8\textwidth]{Server_diagram.drawio.png}
    \caption{Receiving File Metadata Diagram}
\end{figure}
\begin{lstlisting}[language=Python]
# Receiving file metadata
file_metadata = conn.recv(1024).decode()
filename, filesize = file_metadata.split(",")
filesize = int(filesize)
\end{lstlisting}

\subsubsection{Receiving File in Chunks}

\begin{figure}[H]
    \centering
    \includegraphics[width=0.8\textwidth]{receive_file_chunks_diagram.png}
    \caption{Receiving File in Chunks Diagram}
\end{figure}

\begin{lstlisting}[language=Python]
# Receiving file in chunks
received_bytes = 0
with open(f"received_{filename}", "wb") as f:
    progress = tqdm(total=filesize, unit="B", unit_scale=True, desc="Receiving")
    while received_bytes < filesize:
        chunk = conn.recv(1024)
        if not chunk:
            break
        f.write(chunk)
        received_bytes += len(chunk)
        progress.update(len(chunk))
    progress.close()
\end{lstlisting}

\subsection{Client Code Snippet}

\subsubsection{Sending File Metadata}
\begin{figure}[H]
    \centering
    \includegraphics[width=0.8\textwidth]{send_metadata_diagram.png}
    \caption{Sending File Metadata Diagram}
\end{figure}
\begin{lstlisting}[language=Python]
# Sending file metadata
file_metadata = f"{filename},{filesize}"
client_socket.send(file_metadata.encode())
\end{lstlisting}

\subsubsection{Sending File in Chunks}
\begin{figure}[H]
    \centering
    \includegraphics[width=0.8\textwidth]{send_file_chunks_diagram.png}
    \caption{Sending File in Chunks Diagram}
\end{figure}
\begin{lstlisting}[language=Python]
# Sending file in chunks
with open(filename, "rb") as f:
    for chunk in tqdm(iter(lambda: f.read(1024), b""), desc="Sending"):
        client_socket.send(chunk)
\end{lstlisting}


\section{Testing}
The system was tested using:
\begin{itemize}
    \item Small text files (e.g., \texttt{test.txt}).
    \item Large binary files (e.g., images up to 100MB).
    \item Verification tools for checksum (e.g., \texttt{md5sum}).
\end{itemize}

The table below summarizes the results:

\begin{table}[h!]
\centering
\begin{tabular}{|l|c|c|}
\hline
\textbf{File Type} & \textbf{Size} & \textbf{Result} \\ \hline
Text File          & 1KB           & Successful       \\ \hline
Image File         & 10MB          & Successful       \\ \hline
Large File         & 100MB         & Successful       \\ \hline
\end{tabular}
\caption{Testing Results}
\end{table}

\section{Challenges and Solutions}
Challenges included:
\begin{itemize}
    \item Handling large files efficiently.
    \item Ensuring data integrity during transfer.
\end{itemize}
Solutions involved using chunked transfers and MD5 checksums.

\section{Conclusion}
This project demonstrates the implementation of a robust TCP file transfer system. 
It highlights the importance of protocol design and error handling in distributed systems.

\end{document}
